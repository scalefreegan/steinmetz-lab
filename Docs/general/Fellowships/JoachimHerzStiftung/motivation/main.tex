%% start of file `template.tex'.
%% Copyright 2006-2013 Xavier Danaux (xdanaux@gmail.com).
%
% This work may be distributed and/or modified under the
% conditions of the LaTeX Project Public License version 1.3c,
% available at http://www.latex-project.org/lppl/.


\documentclass[11pt,a4paper,sans]{moderncv}        % possible options include font size ('10pt', '11pt' and '12pt'), paper size ('a4paper', 'letterpaper', 'a5paper', 'legalpaper', 'executivepaper' and 'landscape') and font family ('sans' and 'roman')

% moderncv themes
\moderncvstyle{banking}                            % style options are 'casual' (default), 'classic', 'oldstyle' and 'banking'
\moderncvcolor{blue}                                % color options 'blue' (default), 'orange', 'green', 'red', 'purple', 'grey' and 'black'
%\renewcommand{\familydefault}{\sfdefault}         % to set the default font; use '\sfdefault' for the default sans serif font, '\rmdefault' for the default roman one, or any tex font name
%\nopagenumbers{}                                  % uncomment to suppress automatic page numbering for CVs longer than one page

% character encoding
\usepackage[utf8]{inputenc}                       % if you are not using xelatex ou lualatex, replace by the encoding you are using
%\usepackage{CJKutf8}                              % if you need to use CJK to typeset your resume in Chinese, Japanese or Korean

% adjust the page margins
\usepackage[scale=0.8]{geometry}
%\setlength{\hintscolumnwidth}{3cm}                % if you want to change the width of the column with the dates
%\setlength{\makecvtitlenamewidth}{10cm}           % for the 'classic' style, if you want to force the width allocated to your name and avoid line breaks. be careful though, the length is normally calculated to avoid any overlap with your personal info; use this at your own typographical risks...

% personal data
\name{Aaron}{Brooks}
\title{Postdoc@EMBL}                               % optional, remove / comment the line if not wanted
\address{Meyerhofstrasse 1}{69117 Heidelberg}{Germany}% optional, remove / comment the line if not wanted; the "postcode city" and and "country" arguments can be omitted or provided empty
%\phone[mobile]{+1~(234)~567~890}                   % optional, remove / comment the line if not wanted
\phone[fixed]{+49 6221 387-8630}                    % optional, remove / comment the line if not wanted
%\phone[fax]{+3~(456)~789~012}                      % optional, remove / comment the line if not wanted
\email{aaron.brooks@embl.de}                               % optional, remove / comment the line if not wanted
\homepage{aaronbrooks.info}                         % optional, remove / comment the line if not wanted
%\extrainfo{@scalefreegan}                 % optional, remove / comment the line if not wanted
%\photo[64pt][0.4pt]{picture}                       % optional, remove / comment the line if not wanted; '64pt' is the height the picture must be resized to, 0.4pt is the thickness of the frame around it (put it to 0pt for no frame) and 'picture' is the name of the picture file
%\quote{Some quote}                                 % optional, remove / comment the line if not wanted

% to show numerical labels in the bibliography (default is to show no labels); only useful if you make citations in your resume
%\makeatletter
%\renewcommand*{\bibliographyitemlabel}{\@biblabel{\arabic{enumiv}}}
%\makeatother
%\renewcommand*{\bibliographyitemlabel}{[\arabic{enumiv}]}% CONSIDER REPLACING THE ABOVE BY THIS

% bibliography with mutiple entries
%\usepackage{multibib}
%\newcites{book,misc}{{Books},{Others}}
%----------------------------------------------------------------------------------
%            content
%----------------------------------------------------------------------------------
\begin{document}
%-----       letter       ---------------------------------------------------------
% recipient data
\recipient{Joachim Herz Stiftung}{}
\date{August 28, 2015}
\opening{Dear Evaluating Committee,}
\closing{Sincerely,}
%\enclosure[Attached]{curriculum vit\ae{}}          % use an optional argument to use a string other than "Enclosure", or redefine \enclname
\makelettertitle

I am applying for a Joachim Herz Stiftung Add-on Fellowship to supplement an ambitious systems genetics project I have proposed. The project combines longitudinal multiomic molecular profiles with genetics to better understand how Baker's yeast ages. While I have expertise in both bioinformatics and  high-throughput transcriptomics methods, I would like to receive additional training in several emerging technologies to complement this skill set. More specifically, I would use Joachim Herz Stiftung funding to travel to several collaborating labs across the world where I will receive hands-on training in microfluidics, metabolomics, proteomics, and statistical machine learning methods.

I am currently a postdoctoral fellow with Lars Steinmetz at the European Molecular Biology Laboratory (EMBL) in Heidelberg, Germany. The group is internationally recognized for its forward-looking approach to emerging technologies in systems genetics, pioneering a number of technologies widely used in the field. I joined the group to expand my interdisciplinary training, taking on a more complex model system (yeast) and learning how to measure cells at multiple molecular scales. Previously in my PhD, I studied the structure and function of gene regulatory networks (GRNs) in microorganisms. I developed machine learning algorithms to infer GRNs directly from high-throughput gene expression data for two phylogenetically diverse organisms. These genome-scale models were able to predict (accurately and quantitatively) mechanisms responsible for regulation of each gene in the genome (Brooks et al. 2014. Mol Syst Biol). With this project I gained valuable expertise in computational approaches (ensemble learning and graph theory) and several high-throughput experimental methods related to transcription. By the conclusion of my PhD, however, it was clear to me that comprehensive understanding of a biological system would require integrating across multiple scales of biological complexity.

The project I describe in my application emerged from an interest in how technology is reshaping healthcare. I have been fascinated by a new approach to preventative medicine that attempts to quantify and maintain "wellness" by combining longitudinal profiling with genetics.  Here, I saw an opportunity to create a project that would have implications for data integration within the basic sciences as well as personalized medicine. I have therefore designed a project to study a phenotype related to wellness (aging) by combining genetics with longitudinal molecular profiling. My hope is that the project will contribute a data integration paradigm for holistic modeling approaches.

This project will be enriched by the training I will receive through Joachim Herz Stiftung funding. I will use these Add-on funds to conduct extended research stays in laboratories at Stanford, EMBL-EBI, and the University of Luxembourg. Each of these labs has expertise in a technology that will supplement my project. My bigger ambition is to carry some of these collaborations forward to take on other challenging problems in the future.

Thank you for the opportunity to apply for your prestigious award.

\makeletterclosing

\end{document}


%% end of file `template.tex'.
